\documentclass{beamer}

\usepackage{verbatim}
\usepackage{fancyvrb}
\usepackage{amsmath}
\usepackage{mathtools}
\usepackage{booktabs}
\usepackage{amssymb}
\usepackage{graphicx}
\usepackage{calc}
\usepackage{color}
\usepackage{multicol}
\usepackage{wrapfig}
\usepackage{natbib}
\usepackage[ruled,vlined]{algorithm2e}
\usepackage{animate}
\usepackage{mathtools}
\usepackage{listings}
\usepackage{xfrac}

% two col: two columns
\newenvironment{twocol}[4]{
\begin{columns}[c]
\column{#1\textwidth}
#3
\column{#2\textwidth}
#4
\end{columns}
}

\makeatletter
\setbeamertemplate{theorem begin}
{%
\begin{\inserttheoremblockenv}
  {}{\usebeamerfont*{block title}\usebeamercolor[fg]{block title}%
  \inserttheoremname
  %\inserttheoremnumber
  \ifx \inserttheoremaddition \empty \else\ (\inserttheoremaddition)\fi
  \inserttheorempunctuation}
  \normalfont
  }
  \setbeamertemplate{theorem end}{\end{\inserttheoremblockenv}}
\makeatother

\newcommand{\E}{\mathrm{E}}
\newcommand{\Var}{\mathrm{Var}}
\newcommand{\Cov}{\mathrm{Cov}}
\newcommand{\s}{\mathrm{s}}
\newcommand{\Corr}{\mathrm{Corr}}
\newcommand{\rank}{\mathrm{rank}}
\newcommand{\nullspace}{\mathrm{null}}
\newcommand{\myspan}{\mathrm{span}}
\DeclareMathOperator*{\argmax}{arg\,max}
\DeclareMathOperator*{\argmin}{arg\,min}
\DeclareMathOperator*{\softmax}{softmax}

\definecolor{darkgreen}{rgb}{0,0.5,0}

\newtheorem{prop}[theorem]{Proposition}
\newtheorem{exe}{Exercise}
\newtheorem{remark}{Remark}
\newtheorem{myproof}{Proof}

\definecolor{darkgreen}{rgb}{0,0.5,0}

\title{Simultaneous Inference}
\author{Zhenisbek Assylbekov}
\institute{Department of Mathematics}
\date{Regression Analysis}

\AtBeginSection[]
{
  \begin{frame}<beamer>
    \tableofcontents[currentsection]
  \end{frame}
}

\begin{document}

\begin{frame}
  \titlepage
\end{frame}

\begin{frame}{Main Idea}
If a 95\% CI is created for $\beta_0$ and another 95\% CI for $\beta_1$, \pause we cannot say that we are 95\% confident that these two confidence intervals are \textit{simultaneously both} correct.
\end{frame}

\begin{frame}{Controlling the Error Rate}
Let $\alpha$ be a significance level for each of the $g$ tests that we are planning to conduct.\\~\\

\pause Assume we wish to combine all the individual tests into a combined/simultaneous test
$$
\mathrm{H}_0={\mathrm{H}_0}_1\cap{\mathrm{H}_0}_2\cap\cdots\cap{\mathrm{H}_0}_g
$$
\pause $\mathrm{H}_0$ is rejected if any of the ${\mathrm{H}_0}_i$ is rejected.\\~\\

\pause If each ${\mathrm{H}_0}_i$ is conducted independently, then the overall significance level is
\begin{multline*}
\alpha^\ast=\Pr(\text{Reject }\mathrm{H}_0\mid\mathrm{H}_0\text{ is true})=1-\Pr(\text{FTR }\mathrm{H}_0\mid\mathrm{H}_0)\\
=1-\Pr(\cap_{i=1}^g\text{FTR }{\mathrm{H}_0}_i\mid\mathrm{H}_0)=1-\prod_{i=1}^g\Pr(\text{FTR }{\mathrm{H}_0}_i\mid\mathrm{H}_0)=1-(1-\alpha)^g
\end{multline*}
\end{frame}

\begin{frame}{Bonferroni Correction}
If we do not know whether the tests are independent,
\begin{align*}
1-\alpha^\ast&=\Pr(\text{FTR }\mathrm{H}_0\mid\mathrm{H}_0)=\Pr(\cap_{i=1}^g\text{FTR }{\mathrm{H}_0}_i\mid\mathrm{H}_0)\\
&=1-\Pr(\cup_{i=1}^g \text{Rej }{\mathrm{H}_0}_i\mid\mathrm{H}_0)
\ge1-\sum_{i=1}^g\underbrace{\Pr(\text{Rej }{\mathrm{H}_0}_i\mid\mathrm{H}_0)}_{\alpha}\\
&=1-g\alpha\quad\Rightarrow\quad1-\alpha^\ast\ge1-g\alpha\quad\Rightarrow\quad\alpha^\ast\le g\alpha
\end{align*}
\pause Hence, $\alpha^\ast\le g\alpha$, and by choosing individual significance levels as
$$
\boxed{\alpha=\frac{\alpha^\ast}{g}}
$$
for a \textit{given} $\alpha^\ast$ we guarantee that the overall significance level will not exceed $\alpha^\ast$.
\end{frame}

\begin{frame}{Simultaneous Estimation of Mean Responses}
\begin{itemize}
    \item Bonferroni: Can be used for $g$ simultaneous CIs, each with $100\cdot(1-\alpha/g)\%$ confidence level:
    $$
    \hat{Y}\pm t_{1-\alpha/(2g),n-2}\cdot\s[\hat{Y}]
    $$
    \pause If $g$ is large, these intervals will be ``too'' wide for practical conclusions.
    \item\pause Working-Hotelling: A confidence band is created for the entire regression line that can be used for any number of confidence intervals for means simultaneously:
    $$
    \hat{Y}\pm\sqrt{2F_{1-\alpha;2,n-2}}\cdot\s[\hat{Y}]
    $$
\end{itemize}    
\end{frame}

\begin{frame}{Simultaneous Predictions}
\begin{itemize}
    \item Bonferroni: Can be used for $g$ simultaneous PIs, each with $100\cdot(1-\alpha/g)\%$ confidence level:
    $$
    \hat{Y}\pm t_{1-\alpha/(2g),n-2}\cdot\s[b_0+b_1x+\epsilon]
    $$
    \pause If $g$ is large, these intervals will be ``too'' wide for practical conclusions.
    \item Scheff\'e: Widely used method. Like the Bonferroni, the width increases as $g$ increases
    $$
    \hat{Y}\pm\sqrt{g F_{1-\alpha;g;n-2}}\cdot\s[b_0+b_1x+\epsilon]
    $$
\end{itemize}    
\end{frame}

\end{document}